\documentclass{article}
\usepackage{ctex}
\usepackage{geometry}
\geometry{left=3.18cm,right=3.18cm,top=2.54cm,bottom=2.54cm}
\usepackage{graphicx}
\pagestyle{plain}
\usepackage{setspace}
\usepackage{listings}
\usepackage{xcolor}
\usepackage{fontspec}
\usepackage{tikz}
\usepackage{booktabs}
\usepackage{multirow}
\usepackage{hyperref}
\usepackage{ulem}
\setmonofont{Fira Code}
\setmainfont{Times New Roman}
% \setCJKsansfont{Noto Sans CJK SC}
\hypersetup{
    colorlinks=true,
    linkcolor=black,
    filecolor=magenta,      
    urlcolor=cyan
}
\lstset{
  numbers=left, 
  numberstyle=\tiny,
  keywordstyle=\color{blue!70},
  commentstyle=\color{red!50!green!50!blue!50},
  frame=trbl,
  rulesepcolor=\color{red!20!green!20!blue!20},
  basicstyle=\ttfamily\scriptsize
}
\date{}
\begin{document}

	\begin{center}
		\quad \\
		\quad \\
		\heiti \fontsize{30}{17} 大数据处理综合实验
		\vskip 0.5cm
		\heiti \fontsize{45}{17} 课\quad 程\quad 设\quad 计
		\vskip 3.5cm
		\heiti \zihao{2} 网站访问日志分析
	\end{center}
	\vskip 3.5cm

	\begin{quotation}
		\songti \fontsize{15}{15}
		\doublespacing
		\par\setlength\parindent{9em}
		\quad 

		成员:\underline{\makebox[15em]{\qquad xxx \qquad }}
		
		学号:\underline{\makebox[15em]{\qquad xxxxxxxxx \qquad }}
				
		邮箱:\underline{\makebox[15em]{\qquad xxxxxxxxx@xxxxx.xxx.xxx.xx \qquad }}

    成员:\underline{\makebox[15em]{\qquad xxx \qquad }}
		
		学号:\underline{\makebox[15em]{\qquad xxxxxxxxx \qquad }}
				
		邮箱:\underline{\makebox[15em]{\qquad xxxxxxxxx@xxxxx.xxx.xxx.xx \qquad }}
    
    导师: \underline{\makebox[15em]{\qquad xx \qquad }}

    研究领域: \underline{\makebox[13em]{\qquad 云计算与大数据 \qquad }}
		\vskip 2cm
		\centering
		\today
	\end{quotation}
	
	\newpage
	
	\tableofcontents
  
  \newpage
  \section{小组分工}
  
  本次实验的整体代码架构和所用第三方库由两人共同讨论并确定。在确定整体框架后,由xxx负责编写代码,由xxx负责测试代码和编写报告。
  
  \section{课程设计题目}

  我们选择的课程设计题目是第一个,即网站访问日志分析。

  \section{摘要}
  我们采用 MapReduce 框架,完成了网络访问日志分析的任务

  在此课程设计中,我们设计了两套自定义 MapReduce 基类,分别对应解决两类任务所需要的
  不同策略。通过对对应基类进行继承和重载,我们解决了 4 个要求各异的任务。同时,我们利用
  课程中学到的知识,设计了 Combiner 和 Partitioner, 分别用来优化网络传输带宽和自定义中间结果分区,整体解决方案效率较高。
  最终,我们对不同任务输出的结果进行分析,得到了关于此网站状态的有益的信息。

  \section{研究问题背景}
  
  网页日志中包含了一个网页的各种活动信息,对网页日志进行分析,能得出许多深刻而有益的结论。
  随着互联网的飞速发展,各种网站产生的数据飞速增长,复杂度和规模都在不断上升。日志分析是
  典型的可并行化的问题,它包含对每条日志的同样的分析动作,分析产生的综合信息需要后续汇总
  处理,同时,汇总的顺序并不会影响结论,因此非常适合采用 MapReduce 框架设计并行程序
  分析处理。我们的课程设计设计了一套 MapReduce 程序,分析了某网站的网页日志数据,汇总
  出几类信息,并在此基础上得出了关于此网站的一些结论。

  \section{技术难点和解决方案}
  \subsection{主要技术难点}
  在完成实验时,我们遇到了如下一些问题。
  \begin{enumerate}
    \item 日志文件的可靠解析方式
    \item 对任务进行合理分解,并设计 MapReduce 并行化算法
    \item 合理设计代码结构以复用代码并提供未来的可扩展性
  \end{enumerate}
  
  其中的技术难点主要在日志文件的解析和合理设计算法上。日志文件的解析涉及到比较复杂的字符串解析,如解析时间戳、解析URL等等,
  这部分若自己实现,则显得繁琐,所以我们使用了开源的第三方库帮助我们解析日志。对于合理设计算法,难度则主要在设计可复用的代码
  框架并提供可扩展性上,问题本身的可并行性则比较明显。

  下面一节会首先介绍设计思路,然后详细介绍实现细节。
  \subsection{设计思路和详细设计说明}
  \subsubsection{日志解析(任务一)}
  通过观察日志本身和查询资料,我们发现,此日志的格式为\textbf{Combined Log Format}\cite{clf}. 能够解析这种格式的第三方库非常多,我们采用 \verb|grok|\cite{grok}库帮助我们
  直接解析日志文件格式。

  我们自定义 \verb|Log|类以存储日志解析的结果,其内包含多个 \verb|final|字段和对应的 \verb|get|方法。
  我们将其构造函数声明为 \verb|private|, 仅向外提供一个 \verb|parseLog|的静态方法,此方法接受日志字符串
  并返回根据日志构造出的 \verb|Log|对象。这么做之后用户只能通过 \verb|parseLog|构造 \verb|Log|对象,能
  防止构造函数被滥用。

  \verb|parseLog|方法基本上只是对 \verb|grok|库解析日志功能的封装,我们首先使用 \verb|grok|将日志字符串
  解析为键值对,然后利用这些键值对初始化 \verb|Log|对象并返回。

  对于解析出的日志文件,我们仅将其存储在内存中,并不会写入磁盘,后续的任务也不会从磁盘读取任务一
  的中间结果。这是因为,我们写入的解析结果,在磁盘上仍以结构化文本形式存在,后续任务将其读取出来时,
  并不能直接用来初始化对象,仍需要新的一轮字符串解析流程,而且这部分就得由我们自己手工编写了。这样
  我们既没有提升什么效率,又增加了手工编写的工作量以及出错的几率,所以不如直接读取原始日志数据,然后
  采用第三方库将其解析为内存中的数据结构。这样做不仅不会带来性能瓶颈,还能节约工作量,提升可靠性。

  \subsubsection{任务分解和并行算法设计(任务2--5)}

  此课程设计中,任务一已在上一部分解决,剩余五个任务中,前四个可用 MapReduce 解决。这是因为,他们
  都是对每条日志,提取出我们关心的信息输出作为中间结果,然后对中间结果综合形成最终结果。这非常适合
  采用并行程序解决。我们将``从每条日志中提取中间结果''和``综合中间结果''两部分并行化,能带来非常大的
  性能提升。

  我们通过观察几个任务,能大致将任务分为两类。一类是 \verb|Count| 类任务,即我们要做的都是
  统计某种东西出现的个数;另一类是 \verb|Unique| 类任务,即我们不仅统计某种现象的出现,还要对
  现象的来源进行统计,即去重。我们针对这两类任务,分别设计了两套基类,通过对基类的继承和实现,我们
  设计出针对每个任务的算法。

  首先,我们介绍两套基类的实现。

  \paragraph{Count 类任务} 对此类任务,我们在 \verb|CountMapred.java| 文件中定义 \verb|CountMapred| 类来抽象解决方案。

  \verb|CountMapred| 类中包含两个静态抽象类,为 \verb|CountMapper| 和 \verb|CountReducer|,以及一个静态的
  \verb|bindJob| 方法。最后一个方法封装了此类的使用方法。

  两个内部类分别重载了 \verb|map| 和 \verb|reduce| 方法。
  
  \verb|map| 方法的输入键值对为 \verb|<行偏移量, 一行的内容>|。
  输入文件中一行就是一个完整的日志,所以\verb|map|方法可以直接采用这种输入键值对。
  \verb|map| 方法首先调用 \verb|Log| 类的 \verb|parseLog| 静态方法解析日志,然后
  将日志解析结果传给自身的抽象方法 \verb|toKey| 获得需要输出的键。得到这个键以后,\verb|map| 就
  判空并输出键值对\verb|<new Text(key), 1>|。 \verb|map| 方法的输出键值对为 \verb|Text, 1|。其含义为
  我们关心的资源出现了一次。

  \verb|map| 方法所调用的 \verb|toKey| 抽象方法是用来实现自定义策略的地方,使用 \verb|CountMapper| 的
  程序员需要重载此方法,自定义如何从日志中获得感兴趣的信息。此方法可以提供灵活性和可扩展性。

  \verb|reduce| 则比较简单,它将资源出现的次数加起来,然后输出资源出现的总次数。\verb|reduce| 的输入键值对
  类型是\verb|<Text, [value]>|,其意义则是``我们关心的资源和其出现次数列表'',因为中间可能会经过 Combiner,
  所以次数不一定为 1 。\verb|reduce| 的输出则是统计结果,格式为 \verb|<Text, total>|,前者是资源名,后者则是总数。

  我们在 \verb|bindJob| 中做的事,其实和平常直接写 MapReduce 程序时在 \verb|main| 函数
  中做的事相近,此处只是将其封装起来,作为固定的执行策略。我们在这函数中,将 \verb|CountReducer| 
  同时设置为 \verb|Reducer| 和 \verb|Combiner|,因为 \verb|Combiner| 做的事情确实和 \verb|Reducer| 没有区别。
  我们通过这种方式节约了网络传输带宽。

  此类任务并不需要自定义 \verb|Partitioner|。
  
  \paragraph{Unique 类任务} 此类任务的策略定义在 \verb|UniqueMapred.java| 文件中,为 \verb|UniqueMapred| 类。

  此类任务涉及到输入日志中的两个变量,而且涉及到去重,因此实现会更为复杂一些。
  
  在之前的实验中,我们自己定义了 \verb|GenericPair| 类,它可以容纳一对支持 hadoop 接口
  的量,自己也实现了 hadoop 要求的接口,因此可以直接在 MapReduce 的过程中使用。我们
  继承这个类,定义 \verb|TextPair| 类,容纳两个 \verb|Text| 类型的变量。

  我们自定义的 \verb|UniqueMapper| 向外提供两个可重载的抽象方法 \verb|toKey1| 和 \verb|toKey2|,用来
  从日志中获取两个感兴趣的信息,提供灵活性和可扩展性。在 \verb|map| 方法中,我们依旧是首先解析日志,然后调用这两个
  抽象方法获取一对信息,随后输出这一对信息。\verb|map| 的输入格式是 \verb|<偏移量,一行>|, 输出是
  \verb|<TextPair(key1, key2), NullWriteable>|。即输出的值留空,键是一对key。

  我们也定义了 \verb|Combiner| 做本地的 \verb|reduce|,以节省网络带宽。这里我们要做的事很简单,
  因为我们要去重,即对于 \verb|key1| 相同的,合并所有 \verb|key2| 相同的,所以我们
  只需要丢弃 \verb|Combiner| 的 \verb|values|, 输出 \verb|<key, NullWriteable>|即可。
  
  我们自定义 \verb|Partitioner|,以保证相同第一个键的一定到达相同的 \verb|Reducer|。
  自定义方式和课上学过的一样,仅将 \verb|TextPair| 的第一个发给 \verb|hashPartitioner| 即可。

  最后是自定义的 \verb|UniqueReducer|。为了去重,我们在类中保存 3 个变量:\verb|lastK1|、\verb|lastK2|
  和\verb|count|。前两者是记录上一次见到的两个键的内容,最后一个则是在第一个键相同的情况下,
  有多少个不同的第二个键。

  \verb|setup| 函数将 \verb|lastK1| 和 \verb|lastK2| 初始化为空,\verb|count| 初始化为 0 。

  \verb|reduce| 的输入是 \verb|<TextPair, [NullWriteable]>|,此处仅有键是
  有用的。对信息的统计和去重是在 \verb|reduce| 的多次运行中完成的。
  经过我们自定义的 \verb|Partitioner| 和 hadoop 的排序,到达 \verb|Reducer|的
  \verb|Pair| 的顺序必定为具有相同\verb|key1| 的聚在一起,在这聚起来的一堆中,具有
  相同 \verb|key2| 的也聚在一起。因此我们首先将 \verb|key1| 和上一次的比较,如果相同,
  我们就再比较 \verb|key2|和上一次的, 若相同,就不做什么事情,否则更新 \verb|lastK2|,
  并增加 \verb|count|。如果 \verb|key1| 也改变,而且上一次的不是空,说明我们已经统计完
  \verb|key1| 的信息了,就输出,并更新 \verb|lastK1|,最终将其余变量
  更新为 \verb|null| 和 0。这里对算法的介绍未完全介绍判空的情形,实际比这个复杂,详见代码。
  
  \verb|reduce| 的输出是\verb|<key1, count>|,其意义是对于 \verb|key1| 代表的资源,
  互不相同的 \verb|key2| 出现了 \verb|count| 次。

  \verb|reduce| 的 \verb|cleanup| 方法用来输出可能未输出的最后一组信息。

  \verb|UniqueMapper| 同样封装了 \verb|bindJob| 方法,用于封装其使用方式。

  有了这两套基类后,几个任务的实现就简化为继承需要的类,重载方法,在 \verb|main| 函数中初始化并调用
  \verb|bindJob| 三步走了。

  \textbf{任务二} 在 \verb|CountTimes.java|中实现。其继承 \verb|CountMapper|,重载 \verb|toKey| 为从
  \verb|Log| 中获取其 \verb|request| 信息。  
  
  \textbf{任务三} 在 \verb|UniqueIP.java| 中实现。其继承 \verb|UniqueMapper|,重载 \verb|toKey1| 为从
  \verb|Log| 中获取其 \verb|request| 信息,\verb|toKey2| 为获取 \verb|clientIP| 信息。
  
  \textbf{任务四} 在 \verb|CountHour.java| 中实现。其继承 \verb|CountMapper|,重载 \verb|toKey| 为从
  \verb|Log| 中获取精确到小时的时间戳。
  
  \textbf{任务五} 在 \verb|CountTimes.java| 中实现。其继承 \verb|CountMapper|,重载 \verb|toKey| 为从
  \verb|Log| 中获取其 \verb|agent| 信息。

  \section{输入输出格式}
  我们的每个任务的输入都是原始数据,其格式为\textbf{Combined Log Format}\cite{clf}。

  任务的输出格式严格按照课程指导ppt实现,为\verb|key[\TAB]value|形式。

  \section{程序运行试验结果说明和分析}

  按照要求跑出来的任务输出并没有按照值排序,因此我们写了一个很简单的 \verb|Sorter| 将输出的各项以值降序排列,以便看出规律。输出文件可在 hdfs 中找到。

  此网站在两日内有 13770 次访问。用户通常在上午(8时 -- 12时)和傍晚至夜间(19时 -- 00时)访问量较大。在 2013 年 9 月 18 日的下午 15 时至
   17 时,此网站访问量急剧上升,至原来的约1.5倍到两倍,可能是因为其举办了活动或因为一些新闻获得了额外的关注。通过 UA 的统计信息可以看出,此网站的用户使用
   桌面端访问较多,使用 Windows, Macintosh,Linux 的用户数量差距不大,结合后面资源访问信息可看出网站的用户很可能是 IT 从业者。
   UA中还有各类爬虫,说明此网站对爬虫持开放态度,或技术不足。此网站的 UA 信息中单项最多的是\verb|DNSPod-Monitor/1.0|,
   说明其域名解析使用的是DNSPod, 并开启了D监控。

   此网站的资源获取信息中,滤去JavaScript, 图片等杂音,可以从资源中的各类计算机术语看出,这应当是某个IT相关网站,有多个板块,且有博客功能。
   用户大部分时间用在浏览各种文章和博客,当日大概是有一个用户注册。总的来说是比较活跃的社区,不存在关站风险。

  \section{总结}
  本次实验我们采用 MapReduce 并行程序设计对一个网站的日志进行了分析,得到相关信息,并得出了有益的结论。本次课程
  设计的特点是采用 MapReduce 并行程序设计,大大加速了分析日志的速度,并具有较好的性能和可扩展性。
  
  我们的并行程序能够统计\textbf{资源请求次数}、\textbf{独立IP}、
  \textbf{每小时访问网站次数}、\textbf{访问网站的浏览器类型}信息并输出为文件。我们另外
  还实现了一个简单的排序器提高输出文件的可读性。输出的文件仍需要进一步写脚本统计等等,并不是
  非常清晰,可写另外的 MapReduce 过程,将多种统计信息联系起来,能得出更深刻的结论。

  在性能上,我们采用自定义的 \verb|Combiner| 进行本地聚合,降低所需的网络传输带宽,同时采用
  自定义 \verb|Partitioner| 消除了 \verb|reducer| 数据的相关性。但仍存在必须从原始文件
  读入并解析的问题,可以做的改进是在任务一解析好并输出为二进制,在后续任务时,自定义 \verb|InputFormat| 
  直接读二进制初始化结构体,这样能达到更好的性能。

  在可扩展性上,我们针对任务的特点,设计了两组基类并封装了两套处理逻辑,实现任务时,仅需要重载
  对应方法即可,可扩展性较好。基类中仍有不少部分是写死的,在本次设计中是合理的,但若后续还有别的
  任务需要处理,则需要提供更多可重载的抽象接口以提高扩展性。

  \newpage
  \begin{thebibliography}{99}
    \bibitem{clf} \href{https://httpd.apache.org/docs/2.4/logs.html#combined}{https://httpd.apache.org/docs/2.4/logs.html\#combined}
    \bibitem{grok} \href{https://github.com/thekrakken/java-grok}{https://github.com/thekrakken/java-grok}
  \end{thebibliography}
\end{document}
